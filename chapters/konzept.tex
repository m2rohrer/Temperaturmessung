\chapter{Konzept}

\section{Blockschema}

\begin{tikzpicture}
[mybox/.style={rectangle,rounded corners=1mm,minimum
size=6mm,anchor=base,very thick,draw=black!50,top color=white,bottom
color=black!20},mylabel/.style={fill=black!10,font=\small,text
width=25mm},>=stealth]

 \node[mybox](first){Konstantstromquelle}; 
 \node[mybox](sec)[below =of first]{PT1000 oder Präzesionswiderstand für
 Ablgeich}; 
 \node[mybox](third)[below = of sec]{Spannunngsmessung mit
 4-Leiter-Messung};
 \node[mybox](fourth)[below = of third]{Impendanzwandler}; 
 \node[mybox](fourth1)[below = of fourth]{Offsetspannungskompensation}; 
  \node[mybox](fourth2)[below = of fourth1]{Spannungsverstärkung}; 
\node[mybox](five)[below = of fourth2]{18-Bit A/D Wandler};
  \draw[->](first) --(sec);
  \draw[->](sec)--(third);
  \draw[->](third)--(fourth);
  \draw[->](fourth)--(fourth1);
  \draw[->](fourth1)--(fourth2);
 \draw[->](fourth2)--(five);
\end{tikzpicture}




\section{Konstantstromquelle}
Mit Hilfe der Konstantstromquelle wird ein konstanter Strom von 0,1 mA durch den
PT1000 gelassen, um dann mit Hilfe einer 4-Leiter-Messung die Spannung über dem Sensor
zu messen.

\section{PT1000}
Zum Messen der Temperatur wir ein PT1000 verwendet.Platinsensoren sind
hervorragend für die Temperaturmesstechnik geeignet. Alle PT1000-Sensoren haben
1000 Ohm Innenwiderstand. Dieser Wert gilt bei 0°C. Der Temperaturkoeffizient der Sensoren ist einheitlich 3850 ppm/K. 

\section{Impendanzwandler}
Damit die nachfolgende Schaltung keinen Einfluss auf das Messsignal nehmen kann,
wird das Messsignal über eine Impedanzwanlerschaltung mit einem
Operationsverstärker abgekoppelt.

\section{Offsetspannungskompensation}
Da man  den exakten Wert des Stromes berechen will, muss die gemessene Spannung
genau sein und darf keinen Offset des OP's enthalten.

\section{Spannungsverstärkung}
Bei -20°C erhält man eine Spannung von 84,6 mV und bei 80 °C  130,8 mV
eine Spannung von 84.6 mV und

\section{18-Bit A/D Wandler}

Eigentlich würde ein 14-Bit A/D Wandler reichen. Nur haben wir auf dem Markt nur
18 -Bit Wandler gefunden. Wir haben einen mit der Schnittstelle i2C Bus
genommen.


