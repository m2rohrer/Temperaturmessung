\chapter{Aufgabenstellung}
\label{chp:aufgabenstellung}

\subsection*{Thema:	Temperaturmessung}
\begin{tabular}{ll}
Studenten: & 	Martina Rohrer und Anina Schälle\\ 
Betreuer: & 	Guido Keel \\ 
%Partner: & 		Firma Y\\ 
\end{tabular}

\subsection*{Kurzbeschreibung}
In diesem Projekt soll der Analogteil einer Schaltung zur präzisen Temperaturmessung entwickelt werden. Sie wählen elektronische Komponenten, zeichnen Ihre Schaltung, analysieren sie auf Toleranzen und überlegen sich ein Konzept zur Kalibration der Schaltung. 


\subsection*{Aufgabenstellung}
\begin{itemize}
	\item  Es sollen bis zu 5 Sensoren gemessen werden können.
	\item Als Sensoren sollen PT1000 eingesetzt werden.
	\item  Die Sensoren können lange Zuleitungen von bis zu 50m zum Messsystem aufweisen. D.h. es können Störungen der Netzfrequenz auf dem Signal moduliert sein.
	\item Die digitalen Signale können von Ihnen frei definiert werden. Sie können davon ausgehen, dass ein Microcontroller mit einer Taktfrequenz von 20MHz die Signale und digitale Auswertungen übernehmen kann.
	\item Die Schaltung soll kostengünstig sein, d.h. verwenden Sie keine
	superpräzisen, dafür teuren Bauteile.
	\item Sie haben viel Zeit zur Messung, können z.B. die Kalibration der
	Schaltung vorsehen.
\end{itemize}

\subsection*{Erwartete Ergebnisse}
Am Ende des Projektes soll ein dokumentiertes Schema existieren, das mit
geeigneter Ansteuerung aus einem Digitalteil (FPGA oder Microcontroller) die Temperaturen an den Sensoren bestimmen kann.



